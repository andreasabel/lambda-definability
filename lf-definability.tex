\nonstopmode
\documentclass[a4paper]{article}
\usepackage{natbib}
\usepackage[usenames,dvipsnames]{xcolor}
\usepackage{amsmath}

\input{macros}

\newtheorem{theorem}{Theorem}
%\newtheorem*{theorem*}{Theorem}
\newtheorem{proposition}[theorem]{Proposition}
\newtheorem{lemma}[theorem]{Lemma}
\newtheorem{corollary}[theorem]{Corollary}
\newtheorem{conjecture}[theorem]{Conjecture}
\newtheorem{definition}[theorem]{Definition}
\newtheorem{remark}[theorem]{Remark}
\newtheorem{example}[theorem]{Example}

%\newcommand{\qed}{\hfill\ensuremath{\Box}}
% from llncs.cls
%\def\squareforqed{\hbox{\rlap{$\sqcap$}$\sqcup$}}
\def\squareforqed{\ensuremath{\Box}}
\def\qed{\ifmmode\squareforqed\else{\unskip\nobreak\hfil
\penalty50\hskip1em\null\nobreak\hfil\squareforqed
\parfillskip=0pt\finalhyphendemerits=0\endgraf}\fi}

\newenvironment{proof}[1][]{\noindent\ifthenelse{\equal{#1}{}}{{\it
      Proof.}}{{\it Proof #1.}}\hspace{2ex}}{\qed\bigskip}
\newenvironment{proof*}[1][]{\noindent\ifthenelse{\equal{#1}{}}{{\it
      Proof.}}{{\it Proof #1.}}\hspace{2ex}}{\bigskip}

\newcommand{\wk}{\mathord{\uparrow}}
\newcommand{\WW}{\mathbb{W}}
\newcommand{\tAr}{\mathsf{Ar}}
\newcommand{\Ar}[1]{\tAr(#1)}

\title{What is a LF(-definable) function?}
\author{Andreas Abel}
\date{May 2018}

\begin{document}
\maketitle

\section{Functions definable by a STLC term}

\cite{jungTiuryn:tlca93} give a concise description of STLC-definable
functions using Kripke logical predicates.
We recapitulate some of our results in our own notation and
terminology.

\paradot{Syntax}
Types and contexts (snoc-lists of types).
\[
\begin{array}{lllrll}
  \Ty & \ni & S,T,U & ::= & \iota & \mbox{base type}
\\ &&& \mid & U \To T & \mbox{function type}
\\[1ex]
  \Cxt & \ni & \Gamma,\Delta & ::= & \cempty & \mbox{empty context}
\\ &&& \mid & \Gamma.U & \mbox{context extension}
\end{array}
\]
The indexed set of simply-typed variables $\Var \_ \_ : \Cxt \to \Ty \to \Set$
is defined inductively by the
following rules for $x \in \Var \Gamma T$.
\[
  \ru{}{\ind 0 \in \Var {\Gamma.T} T}
\qquad
  \ru{\ind i \in \Var \Gamma T
    }{\ind{i+1} \in \Var {\Gamma.U} T}
\]
The indexed set of simply-typed terms
$\Tm \_ \_ : \Cxt \to \Ty \to \Set$ is defined inductively by the
following rules for $t \in \Tm \Gamma T$.
\begin{gather*}
  \ru{x \in \Var \Gamma T}{x \in \Tm \Gamma T}
\qquad
  \ru{t \in \Tm {\Gamma.U} T}{\lambda t \in \Tm \Gamma {U \To T}}
\\[2ex]
  \ru{t \in \Tm \Gamma {U \To T} \qquad u \in \Tm \Gamma U
    }{t\,u \in \Tm \Gamma T}
\end{gather*}
We write $\Gamma \der t : T$ for $t \in \Tm \Gamma T$.

Let weakening $\wk : \Tm \Gamma T \to \Tm {\Gamma.U} T$ and
single substitution $\_[\_] : \Tm {\Gamma.U} T \to \Tm \Gamma U \to \Tm
\Gamma T$ be defined recursively by the usual generalizations.

\subsection{Interpretation into the full function space}

We fix a set $\denty \iota$ to interpret the base type.  The
interpretation is extended to arbitrary types and contexts by setting:
\[
\begin{array}{llll}
  \denty{U \To T} & = & \denty U \to \denty T
    & \mbox{full (set-theoretic) function space}
\\
  \denty{\cempty} & = & 1
    & \mbox{unit set}
\\
  \denty{\Gamma.U} & = & \denty \Gamma \times \denty U
    & \mbox{cartesian product}
\end{array}
\]
Let $\gamma$ range over elements of $\denty\Gamma$ and $d$ over
elements of $\denty T$.

Variables $\dent\_\_ : \Var \Gamma T \to \denty \Gamma \to \denty T$
are projections defined recursively by:
\[
\begin{array}{rlll}
  \dent {\ind 0} {\gamma,d} & = & d \\
  \dent {\ind{i+1}} {\gamma,d} & = & \dent {\ind i} \gamma \\
\end{array}
\]
This interpretation extends to terms
$\dent\_\_ : \Tm \Gamma T \to \denty \Gamma \to \denty T$
by the following recursive clauses:
\[
\begin{array}{llll}
  \dent{\lambda t}\gamma(d) & = & \dent t {\gamma,d}
   & \mwhere \Gamma.U \der t : T
     \mand d \in \denty U \\
  \dent{t\,u}\gamma & = & \dent t \gamma (\dent u \gamma)
   & \mwhere \Gamma \der t : U \To T \mand \Gamma \der u : U \\
\end{array}
\]
Note that the interpretation satisfies the equational laws of STLC:
\[
  (\beta)\quad
  \denty{(\lambda t)\,u} = \denty{t[u]}
\qquad\qquad
  (\eta)\quad
  \denty{\lambda.\,(\wk t)\,\ind 0} = \denty t
\]

\subsection{Kripke logical predicates}

\cite{jungTiuryn:tlca93} define a rather general notion of Kripke
logical predicate which they call \emph{Kripke logical relation of
  varying arity}.
We will just speak of \emph{Kripke logical predicate}.
% We will not make use of their terminology since we
% think that their \emph{varying arity} aspect is already covered by the
% attribute \emph{Kripke}.
Also, we will use a slight refinement of
their setup due to \cite{fioreSimpson:tlca99}.

We fix a category $\WW$ of worlds and functor $\tAr : \WW \to \Set$,
called the \emph{arity functor}.
Given a world morphism $\tau : w' \to w$, we obtain a function
$\Ar \tau : \Ar{w'} \to \Ar{w}$.  Usually, we will write $\Ar\tau$
just as $\tau : \Ar{w'} \to \Ar{w}$.
%$\tau(a')$ instead of $\Ar(\tau)(a')$ for $a' \in \Ar{w'}$.

A \emph{Kripke logical predicate} (KLP) is a predicate
\[
  \Den\_\_ : (T : \Ty) \to (w : \WW) \to (\Ar w \to \denty T) \to
  \Prop
\]
% (used with set notation $d \in \Den T w a$)
such that the following conditions are met:
\begin{enumerate}
\item Base type monotonicity (contravariant).
  Let $f : \Ar w \to \denty \iota$.  Then\\
  $f \in \Den \iota w$ implies $f \comp \tau \in \Den \iota {w'}$ for all
  $\tau : w' \to w$.
\item Kripke function space. Let $f : \Ar w \to \denty {U} \to
  \denty{T}$.  Then: \\
  $f \in \Den{U \To T} w$ iff $f \comp (\tau,d) \in \Den T {w'}$ for all
  $\tau : w' \to w$ and $d \in \Den U {w'}$.
  Herein, $(f \comp (\tau,d))(a') = f\,(\tau(a'))\,(d(a'))$ for $a'
  \in \Ar{w'}$.
\end{enumerate}
For $\WW$ and $\tAr$ fixed, a Kripke logical predicate is already
determined by its definition at base type $\iota$.

\begin{example}
\label{ex:stlcdef}
A specific KLP to characterize the STLC-definable functions is given
by $\WW = \Cxt$ and $\Ar\Gamma = \denty\Gamma$ and
$f \in \Den\iota\Gamma \iff \exists t \in \Tm \Gamma \iota.\ f =
\denty t$, stating that a function $f : \denty\Gamma \to \denty\iota$
is STLC-definable if it is the interpretation of a term of type
$\iota$ in context $\Gamma$.
\end{example}
\begin{lemma}[Monotonicity at any type]
  If $\tau : w' \to w$ and $f \in \Den T w$ then $f \comp \tau \in
  \Den T {w'}$.
\end{lemma}
A KLP is extended to contexts by setting
\[
\begin{array}{lll}
  \rho \in \Den \cempty w & \iff & \mtrue \\
  \rho \in \Den {\Gamma.U} w & \iff &
    \pi_1 \comp \rho \in \Den \Gamma w \mand
    \pi_2 \comp \rho \in \Den U w
\end{array}
\]
\begin{theorem}[Fundamental theorem of logical relations]
  If\/ $\Gamma \der t : T$ and $\rho \in \Den \Gamma w$ then $\denty t \comp
  \rho \in \Den T w$.
\end{theorem}
\begin{proof*}
By induction on $t \in \Tm \Gamma T$.
\begin{caselist}

\nextcase $\Gamma.T \der \ind 0 : T$.  Yes, since $\denty{\ind 0}
\comp \rho = \pi_2 \comp \rho \in \Den T w$.

\nextcase $\Gamma.U \der \ind{i+1} : T$.
Since $\pi_1 \comp \gamma \in \Den \Gamma w$, by induction hypothesis,
$\denty{\ind i} \comp (\pi_1 \comp \gamma) \in \Den T w$.
Thus $\denty{\ind{i+1}} \comp \gamma \in \Den T w$.

\nextcase $\Gamma \der t\,u : T$ where $\Gamma \der t : U \To T$ and
$\Gamma \der u : U$.
By induction hypothesis, $\denty t \comp \rho \in \Den {U \To T} w$
and $\denty u \comp \rho \in \Den U w$, thus,
$(\denty t \comp \rho) \comp (\tid, \denty u \comp \rho) \in \Den T w$,
which means that $\denty{t\,u} \comp \rho \in \Den T w$.

\nextcase $\Gamma \der \lambda t : U \To T$ where $\Gamma.U \der t : T$.
To show $\denty{\lambda t} \comp \rho \in \Den {U \To T} w$, assume
arbitrary $\tau : w' \to w$ and $d \in \Den U {w'}$.  We have to show
that $f := \denty{\lambda t} \comp \rho \comp (\tau, d) \in \Den T
{w'}$.
Note that
$f(\gamma') = \dent{\lambda t}{\rho(\tau(\gamma'))}(d(\gamma'))
= \dent t{(\rho \comp \tau)(\gamma'),d(\gamma')}$
and $\rho \comp \tau \in \Den \Gamma {w'}$.
With $\rho'(\gamma') := ((\rho \comp \tau)(\gamma'), d(\gamma'))$
have $\rho' \in \Den{\Gamma.U}{w'}$ and thus $\denty t \comp \rho' \in
\Den T {w'}$.
\qed
\end{caselist}
\end{proof*}


\subsection{STLC-definability}

Let a KLP $\Den T \Gamma$ be defined as in Example~\ref{ex:stlcdef}.
\begin{lemma}[Reflection and reification]\bla
  \begin{enumerate}
  \item If $\Gamma \der t : T$ then $\denty t \in \Den T \Gamma$.
  \item If $f \in \Den T \Gamma$ then $f = \denty t$ for some $\Gamma
    \der t : T$.
  \end{enumerate}
\end{lemma}
\begin{proof*}
  By induction on the type.  For base type $\iota$, this is by definition.
  For function type $U \To T$ etc.
\qed
\end{proof*}

\bibliographystyle{abbrvnat}
\bibliography{auto-lf-definability}

\end{document}
